\documentclass[a4paper]{article}

\usepackage[francais]{babel}
\usepackage[utf8]{inputenc}
\usepackage[T1]{fontenc}

\title{ARCSYS2 Projet 1: 7 colors}
\author{Simon Bihel, Corentin Ferry}
\date{March 2016}

\begin{document}
    \maketitle

    \section{Introduction}

    \section{Partie 2}
    % Corentin

    \section{Partie 3}
    % Corentin

    \section{Partie 4}
    % Simon
    \paragraph{4.1.} On choisit un nombre aléatoire en 0 et 7 inclus et on
    ajoute 65 pour arriver sur des caractères entre `A' et `G' inclus.

    \paragraph{4.2.} On parcours le plateau pour savoir qu'elles couleurs sont
    adjacentes au caractère du joueur. Ensuite on choisit aléatoirement une
    couleur et on regarde si elle est valide, sinon on recommence.

    \section{Partie 5}
    % Corentin

    \section{Partie 6}
    % Corentin pour 6.1
    % Simon pour 6.2
    \paragraph{6.2.} Nous sommes directement partis sur un algorithme MinMax
    car c'est un algorithme bien connu. Nous avons pris le score comme
    heuristique. Comme on test chaque couleur pour chaque joueur, la complexité
    est en $O(7^n)$ où $n$ est le nombre de coups prévus. En pratique c'est un
    peu moins car si un coup ne change pas le score du joueur courant alors on
    ne va pas plus loin.  Ensuite nous avons mis en place un algorithme
    AlphaBeta qui rajoute juste quelques conditions pour éviter de parcourir
    des coups qui ne seraient pas mieux.

    \section{Partie 7}
    % Simon
    \paragraph{7.2.} Nous avons utilisé AlphaBeta en utilisant comme
    heuristique la fonction qui calcule l'aire disponible pour le joueur
    adverse. Sur 100 parties contre l'ancien AlphaBeta et avec pour tous les
    deux un nombre des coups prévus de 6, il en gagna 71. À noter qu'il était
    le joueur qui commençait. En inversant les rôles il gagna 56 parties.
    En mettant l'IA hégémonique à la place d'AlphaBeta normal, il gagna 93
    parties.


    \section{Partie 8}
    % Simon
    \paragraph{Meilleur affichage} Nous avons amélioré l'affichage dans le
    terminal en utilisant des couleurs et en affichant par dessus le coups
    précédent coup pour éviter que ça défile. Cela a été fait avec des codes
    d'échappement ANSI.

    \section{Conclusion}

\end{document}
