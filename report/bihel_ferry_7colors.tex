\documentclass[a4paper]{article}

\usepackage[francais]{babel}
\usepackage[utf8]{inputenc}
\usepackage[T1]{fontenc}
\usepackage{listings}
\usepackage{xcolor}
\usepackage{fullpage}


\definecolor{mygreen}{rgb}{0,0.6,0}
\definecolor{mygray}{rgb}{0.5,0.5,0.5}
\definecolor{mymauve}{rgb}{0.58,0,0.82}

\lstset{
  backgroundcolor=\color{white},   % choose the background color; you must add
                                   % \usepackage{color} or \usepackage{xcolor}
  basicstyle=\ttfamily,              % the size of the fonts that are used for
                                   % the code
  breakatwhitespace=false,         % sets if automatic breaks should only happen
                                   % at whitespace
  breaklines=true,                 % sets automatic line breaking
  captionpos=b,                    % sets the caption-position to bottom
  commentstyle=\itshape\color{purple!40!black},    % comment style
  deletekeywords={...},            % if you want to delete keywords from the
                                   % given language
  escapeinside={\%*}{*)},          % if you want to add LaTeX within your code
  extendedchars=true,              % lets you use non-ASCII characters; for
                                   % 8-bits encodings only, does not work with
                                   % UTF-8
  frame=single,                       % adds a frame around the code
  identifierstyle=\color{blue},
  keepspaces=true,                 % keeps spaces in text, useful for keeping
                                   % indentation of code (possibly needs
                                   % columns=flexible)
  keywordstyle=\bfseries\color{green!40!black},       % keyword style
  language=C,                      % the language of the code
  otherkeywords={*,...},           % if you want to add more keywords to the set
  numbers=left,                    % where to put the line-numbers; possible
                                   % values are (none, left, right)
  numbersep=5pt,                   % how far the line-numbers are from the code
  numberstyle=\tiny\color{mygray}, % the style that is used for the line-numbers
  rulecolor=\color{black},         % if not set, the frame-color may be changed
                                   % on line-breaks within not-black text
                                   % (e.g. comments (green here))
  showspaces=false,                % show spaces everywhere adding particular
                                   % underscores; it overrides
                                   % 'showstringspaces'
  showstringspaces=false,          % underline spaces within strings only
  showtabs=false,                  % show tabs within strings adding particular
                                   % underscores
  stepnumber=2,                    % the step between two line-numbers. If it's
                                   % 1, each line will be numbered
  stringstyle=\color{orange},     % string literal style
  tabsize=2,                       % sets default tabsize to 2 spaces
  title=\lstname                   % show the filename of files included with
                                   % \lstinputlisting; also try caption instead
                                   % of title
}

\newcommand{\boardsize}{\ensuremath{\mathtt{BOARD\_SIZE}}}

\title{ARCSYS2 Projet 1: 7 colors}
\author{Simon Bihel, Corentin Ferry}
\date{March 2016}

\begin{document}
    \maketitle

    \section{Introduction}

    \section{Partie 2}
    \paragraph{Question 2.1}
    Nous avons implémenté une fonction permettant de remplir le monde avec
des couleurs aléatoires, représentées par un \texttt{char} dont la valeur est
comprise entre \texttt{0} et \texttt{7}.
Le prototype de cette fonction est le suivant :
\begin{lstlisting}
void fill_board(char* board)
\end{lstlisting}
On utilise dans cette fonction le générateur de nombres aléatoires de la
\texttt{libc}, que l'on suppose préalablement initialisé à l'aide de la
fonction \texttt{srand()}.

    \paragraph{Question 2.2}

    Nous avons implémenté la fonction \texttt{update\_board} sous le prototype 
qui suit :
\begin{lstlisting}
int update_board(char* board, char player, char color)
\end{lstlisting}
Elle procède à la mise à jour du plateau par double boucle, en colorant les 
voisins des cases du joueur \texttt{player} de la couleur \texttt{color}.
Cette opération est répétée jusqu'à ce que plus aucune case du plateau ne soit 
changée.

Compte tenu de la structure de cette fonction, on peut s'assurer qu'elle fait 
bien ce que l'on cherche (autrement dit, que la coloration de proche en proche 
est bien effectuée) en remplissant le plateau d'une seule couleur. Le premier 
joueur à choisir cette couleur devrait alors s'attribuer l'ensemble des cases 
du plateau, sauf celle de départ de son adversaire. 

C'est d'ailleurs ce cas-là qui est le pire relativement à la complexité 
temporelle : c'est celui qui comporte le plus de cases à colorier en un seul 
tour, et surtout, c'est un cas tel qu'il existe une suite de cases adjacentes 
de même couleur entre la case de départ d'un joueur et celle de l'autre, qui 
sont les cases les plus éloignées du plateau.

Le monde va être mis à jour par diagonales successives en partant du joueur qui 
a choisi la bonne couleur, comme sur la figure qui suit. (figure à dessiner)

Alors, il y aura $2(\boardsize-1)$ passes à effectuer, où \boardsize{} est la 
taille du plateau. 

    \paragraph{Question 2.3}
    Nous n'avons pas réécrit la fonction précédente. Néanmoins, on peut 
proposer une implémentation de type \emph{flood-fill}~: il s'agirait de 
procéder récursivement au remplissage des cases selon la fonction récursive 
suivante:
\begin{lstlisting}
fill(i, j, couleur_cible, joueur) :
   - Si couleur(i, j) == joueur ou couleur(case) == couleur_cible:
        changer_couleur(i, j, couleur_cible)
        fill(i+-1, j, couleur_cible)
        fill(i, j+-1, couleur_cible)
   - Sinon, ne rien faire
\end{lstlisting}
    \section{Partie 3}
    \paragraph{Question 3.1} Compte tenu du fait que nous allions devoir 
écrire des stratégies de jeu automatiques, nous avons écrit la fonction 
\texttt{game()} dont la structure permet de choisir la méthode de jeu de chaque 
joueur. Les fonctions de stratégies auxquelles \texttt{game()} fait appel 
renvoient toutes un \textt{char} qui est la couleur choisie par le joueur.
Il existe donc une fonction \texttt{ask()}, dont le rôle est de demander à 
l'utilisateur la couleur qu'il a choisi de jouer. 

    \section{Partie 4}
    % Simon
    \paragraph{4.1.} On choisit un nombre aléatoire en 0 et 7 inclus et on
    ajoute 65 pour arriver sur des caractères entre `A' et `G' inclus.

    \paragraph{4.2.} On parcours le plateau pour savoir qu'elles couleurs sont
    adjacentes au caractère du joueur. Ensuite on choisit aléatoirement une
    couleur et on regarde si elle est valide, sinon on recommence.

    \section{Partie 5}
    % Corentin

    \section{Partie 6}
    % Corentin pour 6.1
    % Simon pour 6.2
    \paragraph{6.2.} Nous sommes directement partis sur un algorithme MinMax
    car c'est un algorithme bien connu. Nous avons pris le score comme
    heuristique. Comme on test chaque couleur pour chaque joueur, la complexité
    est en $O(7^n)$ où $n$ est le nombre de coups prévus. En pratique c'est un
    peu moins car si un coup ne change pas le score du joueur courant alors on
    ne va pas plus loin.  Ensuite nous avons mis en place un algorithme
    AlphaBeta qui rajoute juste quelques conditions pour éviter de parcourir
    des coups qui ne seraient pas mieux.

    \section{Partie 7}
    % Simon
    \paragraph{7.2.} Nous avons utilisé AlphaBeta en utilisant comme
    heuristique la fonction qui calcule l'aire disponible pour le joueur
    adverse. Sur 100 parties contre l'ancien AlphaBeta et avec pour tous les
    deux un nombre des coups prévus de 6, il en gagna 71. À noter qu'il était
    le joueur qui commençait. En inversant les rôles il gagna 56 parties.
    En mettant l'IA hégémonique à la place d'AlphaBeta normal, il gagna 93
    parties.


    \section{Partie 8}
    % Simon
    \paragraph{Meilleur affichage} Nous avons amélioré l'affichage dans le
    terminal en utilisant des couleurs et en affichant par dessus le coups
    précédent coup pour éviter que ça défile. Cela a été fait avec des codes
    d'échappement ANSI.

    \section{Conclusion}

\end{document}
